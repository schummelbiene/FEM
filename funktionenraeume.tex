\section{Funktionenräume zur schwachen Formulierung elliptischer RWA}
'Einfache' elliptische RWA: 
\begin{align*}
- \D u &= f \text{ in } \Omega \\
u &= 0 \text{ auf } \partial \Omega
\end{align*}
Dabei ist $\Omega \subset \R^2, \R^3$ und offen, beschränkt und zusammenhängend. 
Was verstehen wir unter eine Lösung dieses Problems? Klassischerweise sollte $u \in C^2(\Omega) \cap C(\bar \Omega)$ sein. Leider existiert so eine klassische Lösung nicht immer. Im eindimensionalen existiert sie, im zweidimensionalen sind wir schon auf die Bedingungen angewiesen. 

Ursachen: komplizierter Rand, $f$ ist unstetig

Ein Ausweg ist ein neuer Lösungsbegriff: \emph{schwache Lösungen}

Grundidee: im Eindimensionalen: $-u'' = f$, $u(0)= u(1) = 0$
\begin{align*}
  - \int_0^1 u'' v = \int_0^1 f v  
\end{align*}
nun partielle Integration (2D: Integransatz)
\begin{align*}
  \left. - u'v \right|_0^1 + \int_0^1 u'v' = \int_0^1 fv
\end{align*}
$v$ genüge den Randbedingungen
\begin{align*}
  \Rightarrow \int_0^1 u'v' = \int_0^1 fv
\end{align*}
genannt: \emph{Variationsgleichung}. Mit der Forderung $u,v \in H_0^1(\Omega)$ (einem Sobolevraum) werden $u$ und $v$ zu schwachen Lösungen.

Ziel der Sobolevräume: Die Integrale der Variationsgleichungen sollen Sinn ergeben und nette Eigenschaften haben. 

$L^p$-Räume:
\begin{align*}
  L^p(\Omega) \coloneqq \{ f: \Omega\rightarrow \bar \R \int_\Omega \norm{f}^p < \infty, p \in[1, \infty) \}
\end{align*}
Für $p = \infty$:
\begin{align*}
  L^\infty(\Omega) \coloneqq \{ f: \Omega\rightarrow \bar \R \esssup f <\infty) \}  
\end{align*}
$L^p$-Räume sind Banachräume, für $p = 2$ Hilberträume.
\begin{align*}
  \norm{f}_{L^p}= \{\int_\Omega \norm{f}^p\}^{1/p}, \norm{f}_{L^\infty}= \esssup f
\end{align*}
Skalarprodukt im $L^2$: $(f,g) = \int_\Omega fg$
\begin{lemma}
In Prähilberträumen (Räume mit Skalarprodukt) gilt: 
Cauchy-Schwarz: $\norm{(f,g)} \leq \norm{f} \norm{g}$
\end{lemma}
\begin{beweis}
  \begin{align*}
    &(f + \lambda g, f + \lambda g) \geq 0, \lambda \in \R\\
&(f, f) + 2(f,g)\lambda+ (g,g)\lambda^2 \geq 0
  \end{align*}
\end{beweis}

Wir benötigen als nächstes einen angepassten Ableitungsbegriff. Sei dazu $u \in C^1(\bar \Omega)$, $v \in C^\infty(\Omega)$. Dann gilt (partielle Integration)
\begin{align*}
  \int_\Omega \frac{\partial u}{\partial x_j} v = \int_{\partial \Omega}u v \cos (n, e^j)- \int_\Omega u \frac{\partial v}{\partial x_j}
\end{align*}
mit dem äußeren Normalenvektor $n$ (bezüglich $\partial \Omega$) und dem $j$-ten Einheitsnormalenvektor $e^j$. 
Sei speziell $v \in C^\infty_0 (\Omega)$. Dann erhalten wir 
\begin{align*}
  \int_\Omega \frac{\partial u}{\partial x_j} v = -\int_\Omega u \frac{\partial v}{\partial x_j}
\end{align*}

\begin{definition}
  Sei $u$ messbar. Falls eine messbare Funktion $w$ exisitert mit 
  \begin{align*}
    \int_\Omega w v = - \int_\Omega u \frac{\partial v}{ \partial x_j},
  \end{align*}
so heißt $w$ \emph{schwache Ableitung} von $u$ nach $x_j$.
\end{definition}

\begin{beispiel}
  Sei $\Omega = (0,1)$ und $a \in (0,1)$.
  \begin{align*}
    f(x) =
    \begin{cases}
       f_1(x) &0<x<a \\
       f_2(x) &a<x<1
    \end{cases}
  \end{align*}
$f_1$ und $f_2$ seien klassisch differentierbar. Ist $f$ schwach differentierbar?

\begin{align*}
  \int_0^1 f v' &= \int_0^a f_1 v' + \int_a^1 f_2 v'\\ 
&= f_1 v |_0^a + f_2 v |_a^1 - \int_0^a f_1'v - \int_a^1 f_2'v \\
&=(f_1(a)-f_2(a))v(a)- \dots = \int_0^1 wv
\end{align*}
Ergebnis: $f$ besitzt genau dann eine schwache Ableitung, wenn $f_1(a)= f_2(a)$, diese ist 
\begin{align*}
    w =
    \begin{cases}
       f_1'(x) &\text{ in } (0,a) \\
       f_2'(x) &\text{ in } (a,1)
    \end{cases}  
\end{align*}
\end{beispiel}

Fundamentale Erkenntnis:
\begin{itemize}
\item Notwendig für schwache Differentierbarkeit ist Stetigkeit
\item stetige Funktionen, die stückweise klassisch differentierbar sind, sind schwach differentierbar (auch mehrdimensional)
\end{itemize}

Beispiel: 'übliche' finite Elemente: stetig, stückweise polynomial

Analog zur Definition werden jetzt Ableitungen höherer Ordnung definiert. Schreibweise: $\partial^\alpha \coloneqq \frac{\partial^{\norm{\alpha}}}{\partial x_1^{\alpha_1} \partial x_2^{\alpha_2} \dots \partial x_n^{\alpha_n}}$ mit $\norm{\alpha} = \sum \alpha_i$ und $\alpha_i \geq 0$, ganz.

\begin{definition}
  Sei $u$ messbar und es exisitere ein messbares $w$ mit 
  \begin{align*}
    \int_\Omega w v = (-1)^{\norm{\alpha}} \int_\Omega u \partial^\alpha v
  \end{align*}
für alle $v \in C^\infty_0(\Omega)$. Dann heißt $w$ \emph{schwache Ableitung} der Ordnung $\norm{ \alpha}$ von $u$, $w = \partial ^\alpha u$ im schwachen Sinne.
\end{definition}
\paragraph{Sobolevräume} von Sobolev, 1935 + x 
$H^1 (\Omega)$: Raum aller quadratischen Funktionen, deren schwache erste Ableitungen quadratisch integrierbar sind.
\begin{align*}
  \norm{u}_1 = \left \{ \int_\Omega u^2 + \int_\Omega \sum_{i= 1}^n \left( \frac{\partial u}{\partial x_i}\right)^2 \right \}
\end{align*}
Hilbertraum (vollständig!)

%\datum{14. Oktober 2014}
\begin{align*}
  H^1(\Omega) \coloneqq \{ \sigma: \sigma \in L_2, \nabla \sigma \in L_2\}
\end{align*}
mit der Norm 
\begin{align*}
  \norm{\sigma}_1 = \left( (\sigma, \sigma) + (\nabla \sigma, \nabla \sigma)\right)^{0.5}
\end{align*}
Bemerkung zum $C^1(\bar \Omega)$ Normierung? 
\begin{enumerate}
\item $\norm{v} \coloneqq \max \norm{v} + \max \norm{\nabla v} \,  \Rightarrow$ Banach-Raum, aber es existiert kein Skalarprodukt mit $(v, v) = \norm v ^2$
\item $\norm v \coloneqq \left( (v,v) + (\nabla v, \nabla v)\right)^{0.5}$: normierter Raum mit Skalarprodukt: $(u,v)_1 = \int_\O uv + \int_\O \nabla u \nabla v$. Raum ist nicht vollständig, also kein Hilbertraum. Vervollständigung dessen führt auf $H^1$.
\end{enumerate}

Allemeine Definition von Sobolev-Räumen:
\begin{align*}
  W_p^l (\O) = \{ v \in L^p: \partial^\alpha v \in L^p \text{ für } \norm \alpha \leq l \}
\end{align*}
mit $p \in [1, \infty], l = 1, 2, \dots $. Fall $p = 2: W_p^l = H^l$.
Oberer Index: Wie viele Ableitungen existieren im schwachen Sinne?

Unterer Index: 

$W_p^l$ ist Banachraum, für $p = 2$ Hilbertraum.

Norm in $W_p^l(\O)$:
\begin{align*}
  \norm v_{W_p^l} \coloneqq \left\{\int_\O \sum_{\norm{\alpha}\leq l} \norm{\partial^\alpha v}^p \right\} ^{1/p}
\end{align*}
\begin{bemerkung} (Meyers, Serrin 1964)
Für $1 \leq p < \infty$ ist $C^\infty(\O) \bigcap W_p^l(\O)$ (bel. oft differentierbar und endliche Norm) dicht im $W_p^l(\O)$.
Anderer Zugang statt verallgemeinerte Ableitung. Hilfreich, wenn man Ungleichungen beweisen will.

 Wenn man in einem Sobolevraum eine Ungleichung beweisen will: beweise für hinreichend glatte Funktionen, und dann argumertiere, dass diese dicht liegen und folgere dann für andere Funktionen. 
\end{bemerkung}
Wir fragen uns: Sind $H^1$-Funktionen stetig?

\paragraph{Fall $n=1$:} $\O = (0,1)$ (wende an $(a+b)^2 = \leq 2(a^2 + b^2)$) und Cauchy-Schwarz für Integrale: 
\begin{align}
  v(x) &= v(y) + \int_y^x v'(t) dt \notag \\
  v^2(x)& \leq 2 \left(v^2(y) + \left(\int_y^x v'\right)^2\right) \notag\\
  v^2(x)& \leq 2 \left(v^2(y) + \norm{x-y} \left(\int_y^x v'^2\right)\right) \notag \\
v^2(x)& \leq 2 \left(v^2(y) +  \left(\int_0^1 v'^2\right)\right) \quad \to \int_0^1 dy \notag\\
v^2 &\leq 2 (\int_0^1 v^2 + \int_0^1 v'^2) \label{eq_star}
\end{align}
Also sind im Eindimensionalen die $H^1$-Funktionen beschränkt. Weiter:
\begin{align}\label{eq_2star}
  \norm{v(x)- v(y)} \leq \sqrt{x-y} \left( \int_0^1 v'^2 \right) ^{0.5}
\end{align}
Arzela-Ascoli: Cauchyfolge aus $H^1 \cap C^\infty$: \eqref{eq_star} impliziert Beschränktheit, \eqref{eq_2star} gleichgradige Stetigkeit, die Grenzfunktionen sind stetig!

\paragraph{Fälle 2D oder 3D} $H^1$-Funktionen können unbeschränkt sein.
\begin{beispiel} 2D:

$\O$ = EInheitskreis, Polarkoordinaten
\begin{align*}
  u(x, y) \coloneqq \ln \ln \frac{2}{\sqrt{x^2 + y^2}} = \ln \ln \frac{2}{r}
\end{align*}
$u$ ist unbeschränkt! Aber: $u \in H^1$. Es ist zu zeigen: $\int u^2< \infty$

\begin{align*}
  \int (u_x^2 + u_y^2)&<\infty \\
( u_x^2 + u_y^2 &= u_r^2 + r^2u_\phi^2 ) \\
\int_{EK} \frac{1}{(\ln 2/r)^2} r^{-2} r dr \; d \phi &= 2 \pi \int_0^1 (\ln 2/r)^{-2} r^{-1} dr\\
&= 2 \pi (-) \int_\infty^{\ln 2} t^{-2} dt \\
&< \infty
\end{align*}
(verwende $\ln 2/r = t$ und $-r^{-1} dr$ = dt). Das ist endlich!
\end{beispiel}
\begin{beispiel} 3D
Einheitskugel, $u = r^{-\alpha}$, $0< \alpha<0.5$ ist nicht beschränkt, aber $u \in H^1$.
\end{beispiel}
Warum ist das ärgerlich? Die Lösungen elliptischer RWA zweiter Ordnung liegen im Allgemeinen im $H^1$. Bei der Mehtode der finiten Elemente ist es üblich, den Fehler durch einen Interpolationsfehler abzuschätzen. Für $H^1$-Funktionen ist die 'normale' Interpolierende \emph{nicht} definiert.
\begin{beispiel}
  erste schwache Formulierung einer elliptischen RWA
  \begin{align*}
    - \Delta u + c u &= f, \qquad c \in L^\infty, f \in L^2\\
\frac{\partial u}{\partial n} &= 0 \qquad \partial \O
  \end{align*}
(homogene Neumannbedingungen)
\end{beispiel}
\begin{align*}
  -\int_\O (\Delta u ) v + \int_\O c u v = \int_\O fv\\
  -\int_{\partial\O} \frac{\partial u}{\partial n} v + \int_{\O} \nabla u \nabla v \int_\O c u v = \int_\O fv 
\end{align*}
gesuct ist $u \in H^1(\O)$ mit 
\begin{align*}
   \int_{\O} \nabla u \nabla v \int_\O c u v = \int_\O fv\\
\end{align*}
für alle $v \in H^1(\O)$.
Zu möglichen Randbedingungen: 
\begin{itemize}
\item Dirichletbedingungen (1. Art): $u = \phi $auf$ \partial \O$
\item Neumannbedingungen (2. Art): $\frac{\partial u}{\partial n} = \psi$ auf $\partial \O$
\item Robinbedingungen (3. Art): $\frac{\partial u}{\partial n} + \alpha u = \chi$ auf $\partial \O$
\end{itemize}
Es scheint unklar, wie man Dirichletbedingungen behandeln kann. 

Hier helfen \emph{Spursätze} (RB heißen auch Spuren).
\begin{lemma} Spurlemma

$\O$ besitze einen Lipschitzrand und sei beschränkt (wie meistens). Dann ist die Abbildung 
\begin{align*}
  C^1(\bar \O) \cap W^1_p(\O) \to L^p(\partial \O)
\end{align*}
 (Funktionen mit endlicher Norm, die auf dem Rand existieren) linear und beschränkt, das heißt,
 \begin{align} \label{eq:spur}
   \norm u _{L^p(\partial \O)} \leq c \norm u_{W_p^1(\O)} \qquad \forall u \in C^1(\bar \O) \cap W^1_p (\O)
 \end{align}

\end{lemma}

\emph{L-Rand}: $\forall x \in \partial \O$ existiert eine Umgebung, in der der Rand der Graph einer L-stetigen Funktion ist. 

\begin{beispiel} Spursatz gilt nicht

$\O = \{0 < y < x^5 < 1\}$, $u = x^{-1}$, $p = 2$.

$u$ ist in $H^1$, die Spur auf dem Rand liegt aber nicht im $L^2$.
\begin{itemize}
\item $\int_0^1 x^{-2}$ exisitert nicht (Spur auf dem Rand nicht in $L^2$)
\item $u \in H^1$: nachrechnen
\end{itemize}
\end{beispiel}

\begin{beweis} Beweis des Spurlemmas für ein Rechteck

Untere Seite des Einheitsquadrats ($= \Gamma_1$):
\begin{align*}
    u(x,0) &= u(x,y) + \int_y^0 u_y(x,t) dt\\
    u^2(x,0) &\leq 2 \left( u^2(x,y) + \int_0^1u^2_y (x,y) dt\right)\\
      \int_{\Gamma_1} u^2 &\leq 2 \left( \int_0^1u^2(x,y) dx + \int_{\O}u^2_y\right)\\
   \int_{\Gamma_1} u^2 &\leq 2 \left( \int_\O u^2 + \int_{\O}u^2_y\right)
  \end{align*}
\end{beweis}
\begin{folgerung}
  Die Abbildung $\gamma: H^1(\O) \cap C^1(\bar \O) \to L^2(\partial \O)$ ist auf einer dichten Menge im $H^1$ eine beschränkte, lineare Abbildung. Sie kann damit unter Beibehaltung von \eqref{eq:spur} auf den $H^1$ erweitert werden:
  \begin{align*}
    \norm{\gamma(u)}_{L^2(\partial \O)} \leq C \norm{u}_{1,\O} \qquad \forall u \in H^1(\O)
  \end{align*}
$\gamma(u)$ heißt \emph{Spur} oder \emph{Randwert} von $u \in H^1$ auf $\partial \O$.
\end{folgerung}

\begin{definition}
  \begin{align*}
    H_0^1(\O) = \overline{(C^\infty_0)}^{H_1}
  \end{align*}
ist der Abschluss von $C_0^\infty$ bezüglich der $H^1$-Norm.
\end{definition}
\begin{bemerkung}
  Nicht jede $L^2$-Funktion auf dem Rand ist Randwert einer Funktion aus dem $H^1$. Raum der Randwerte von $H^1$-Funktionen ($H^{1/2}(\partial \O)$) ist echter Teilraum des $L^2(\partial \O)$ 
\end{bemerkung}
\begin{definition}
  $  H^{-1} (\O)  = $ Dualraum zum $H_0^1(\O)$
(Menge aller linearen Funktionale auf $H_0^1$)
\end{definition}

D-Bedinungen: $-\Delta u = f, f \in L^2$

$u = \phi$ auf $\partial \O$, Vorraussetzung $\phi \in H^{1/2}(\partial \O)$. Es existiert eine Funktion $u_0 \in H^1$ mit $f u_0 = \phi$
Schwache Formulierung:
\begin{align*}
  u-u_0 \in H_0^1(\O)\\
(\nabla u, \nabla v)= (f, v) \quad \forall v\in H_0^1 (\O)
\end{align*}

Betrachte 
\begin{align*}
  -\Delta u = f \text{ in } \O\\
u =
\begin{cases}
  1 & \Gamma_1\\
0 & \Gamma_2
\end{cases}
\end{align*}
Solche unstetigen Randfunktionen können nicht entstehen. ($\phi \not \in H^{1/2}(\partial \O)$)

Für $V$-Elliptizität brauchen wir nun die \paragraph{Friedrich'sche Ungleichung}
\begin{itemize}
\item 1D: $\O = (0,1)$: $v(x) = \int_0^x v'$ mit der Vorraussetzung $v(0) = 0$. Mit Cauchy-Schwarz erhalten wir
\begin{align*}
  v^2(x) \leq \_0^1 v'^2
\end{align*}
Das ist schon die Friedrich'sche Ungleichung. Sie gilt für alle $v \in H^1_0$.
\item 2D: $\O = (0, l)^2$ mit der Vorraussetzung $v \in H_0^1 (\O)$
  \begin{align*}
    v (x, y) = v(0,y) + \int_0^x v_x(t, y) dt\\
    v^2 (x, y) \leq l \int_0^l v^2_x(t, y) dt\\
   \int_\O v^2 \leq l^2 \int_\O v^2_x
  \end{align*}
\item allgemeine Friedrich'sche Ungleichung: $\O \subset \R^n$ sei einthalten in eienm Würfel der Kantenlänge $l$. Dann gilt
  \begin{align*}
    \norm v _{L^2 (\O) }  \leq l(\int_\O(\nabla v)^2)^{1/2} \quad \forall v \in H^1_0 (\O)
  \end{align*}
\end{itemize}

Erste Anwendung: Charakterisierung gewisser Seminormen als äquivalent zu Normen
\begin{align*}
H^1: \quad \nnorm v _1^2 &= (v,v) + (\nabla v, \nabla v)\\
 &= \nnorm v_0^2 + \norm v_1^2
\end{align*}
(Seminorm). Folgerung von Friedrichs: Im $H_0^1 (\O)$ sind $\nnorm \cdot _1$ und $\norm \cdot _1$ äquivalente Normen.

\begin{itemize}
\item $\norm v _1 \leq \nnorm v_1$
\item $\exists c : \nnorm{v}_1^2 \leq c \norm v _1^2$ ?

$\nnorm{v}_0^2 + \norm v_1^2  \leq c \norm v _1^2$ ?

$\nnorm{v}_0^2 \leq (c-1) \norm v _1^2$ ?

Ja, nach Friedrich'scher Ungleichung für $v \in H^1_0$.
\end{itemize}

Analog: $H^2: \nnorm v_2^2 = \nnorm v_0^2 + \norm v_1^2 + \norm v_2^2$

$H_0^2 \coloneqq \overline{C_0^\infty}^{H^2}$, $\norm \cdot_2$ ist Norm auf $H_0^2$.

Weitere Ungleichungen vom Poincare-Friedrichs-Typ:

$\O$ besitze einen L-stetigen Rand. 

$\O_1 \subset \O$ sei ein Teilgebiet von $\O$ mit positivem Maß im $\R^n$.

$\Gamma_1 \subset \partial \O$ sei ein Teilrand mit positivem Maß im $\R^{n-1}$.
Dann gelten die folgenden Ungleichungen für $u \in H^1(\O)$:
\begin{align*}
\nnorm u_0^2 \leq c  \left(\norm u_1^2 + \norm{\int_{\O_1} u}^2 \right)\\
\nnorm u_0^2 \leq c  \left(\norm u_1^2 + \norm{\int_{\Gamma_1} u}^2 \right)
\end{align*}

(siehe Gajewski, Gröger, Zacharias: 'Nichtlineare Operatorgleichungen', Lemma 1.36)

\begin{definition}
  $U$ und $V$ seinen normierte Räume. $U$ heißt stetig eingebettet in $V$, wenn $U \subset V$ und eine Konstante $c$ exisitert mit 
  \begin{align*}
    \nnorm u_V \leq c \nnorm u_U
  \end{align*}
Schreibweise $U \emb V$.
\end{definition}
Beispiel: $H^1$ ist stetig eingebettet in $L_2$. 
$C^{j, \beta} (\bar \O)$: klassisch $j$-mal stetig differentierbar, $j$-te Ableitung $\beta$-hölderstetig ($0 <\beta \leq 1$).

\begin{satz}
  $\O$ besitze einen L-stetigen Rand, $\O \subset \R^n$. Sei $0 \leq j \leq k$, $1 \leq p < \infty$ und $0<\beta <1$. Dann gilt für
  \begin{align*}
    k-j-\beta> \frac{n}{p}
  \end{align*}
die Einbettung 
\begin{align*}
  W_p^l (\O) \emb C^{j, \beta} (\bar \O).
\end{align*}
(Adam: 'Sobolev spaces')
\end{satz}
Spezialfälle: 
\renewcommand{\labelenumi}{(\theenumi)}
\begin{enumerate}
\item $p = 2$, $j = 0$: $k -  \beta > n/2$

$n = 1$: $k > 1/2 + \beta \quad$ $H^1$-Funktionen sind stetig. 

$n = 2$: $k > 2/2 + \beta \quad$ $H^2$-Funktionen sind stetig. 

$n = 3$: $k > 3/2 + \beta \quad$ $H^2$-Funktionen sind stetig. 
\end{enumerate}
\begin{folgerung}
  Für $n = 2, 3$ kann man aus zu Funktionen aus dem $H^2(\O)$ normale Interpolierende definieren!
\end{folgerung}