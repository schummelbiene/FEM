%\datum{23. Oktober 2014}
\section{Ritz-Galerkin-Verfahren}
\begin{align*}
  a(u, v )= f(v) \quad \forall v \in V
\end{align*}
Die Vorraussetzungen des Lax-Milgram-Lemmas seien erfüllt. $V_h \subset V$ sei ein \emph{endlichdimensionaler} Teilraum von $V$. 

Galerkin-Verfahren (1915): Gesucht ist ein $u_h \in V_h$ mit 
\begin{align*}
  a(u_h, v_h) = f(v_h) \quad \forall v_h \in V_h
\end{align*}
Das Schöne an dem Ganzen: das diskrete Problem besitzt eine Lösung (nach dem Lax-Milgram-Lemma), und Stabilität des diskreten Problems auch:
\begin{align*}
  \nnorm{u_h} \leq \alpha^{-1} \nnorm f _*
\end{align*}
Ritz-Verfahren: $J(v) \to \min$ für $v \in V$. Sei $V_h \subset V$:
\begin{align*}
  J(v_h) \to \min \quad \forall v_h \in V_h
\end{align*}
Speziell: $a(\cdot, \cdot)$ symmetrisch, $J(v) = 0.5 a(v, v) - f(v)$.
Dann folgt Ritz = Galerkin.

Zunächst folgt für das Galerkin-Verfahren:
\begin{align*}
  a(u-u_h, v_h)= 0 \quad \forall v_h \in V_h,  
\end{align*}
die sogenannte \emph{Galerkin-Orthogonalität}.
\begin{lemma} Cea-Lemma
  Die Vorraussetzungen von Lax-Milgram seien erfüllt. Dann gilt:
  \begin{align*}
    \nnorm{u-u_h} \leq M/\alpha \inf_{v_h \in V_h} \nnorm{u- v_h}
  \end{align*}
\end{lemma}
  \begin{beweis}
\begin{align*}
  \alpha\nnorm{u-u_h}^2 &\leq a(u- u_h, u-u_h) \\
&= a(u-u_h, u-v_h) + a(u-u_h, v_h - u_h)\\
&= \leq M \nnorm{u-u_h}\nnorm{u-v_h} + \nnorm{u-u_h, v_h - u_h}\\
\nnorm{u-u_h} &\leq M/\alpha \nnorm{u-v_h}
\end{align*}
\end{beweis}
Der Fehler des Galerkinverfahrens ist also proportional zum Fehler der Bestapproximation von $u$ in $V_h$. Das Galerkinverfahren ist \emph{quasi-optimal}.
\begin{bemerkung} zur Babuska-Necas-Theorie

  \begin{align*}
    a(u, v)& = f(v) \quad \forall v \in V\\
u \in W:& \\
&\Downarrow \text{Diskretisierung}\\
u_h \in W_h &\subset W: a(u_h, v_h) = f(v_h) \forall v_h \in V_h \subset V \text{ (Petrov-Galerkin-Verfahren)}
  \end{align*}
  Beim Petrov-Galerkin-Verfahren ist Ansatz- und Testraum nicht gleich. Bedingung für die Existenz der Lösung nich ohne Weiteres auf diskreten Fall übertragbar. 
  \begin{itemize}
  \item $\dim W_h = \dim V_h$\\
  \item $a(\cdot , \cdot)$ beschränkt auf $W_h \times V_h$
  \item $\sup_{V_h}\frac{a(w_h, v_h)}{\nnorm{v_h}}\geq \alpha_h \nnorm{w_h}$, $\alpha_h > 0 $ (inf-sup-Bedingung)
\item $a(w_h, v_h) = 0$ für alle $v_h \Rightarrow w_h = 0$
  \end{itemize}
Daraus folgt die Existenz von $u_h$.
Unter diesen Vorraussetzungen kann man ebenfalls ein Cea-Typ-Lemma beweisen: 
\begin{align*}
  \nnorm{u_h - v_h} \leq \frac 1 \alpha \sup \frac{a(u- w_h, v_h)}{\nnorm {v_h}} \leq \frac M {\alpha_h} \nnorm{u-w_h}
\end{align*}
also
\begin{align*}
  \nnorm{u-u_h} &\leq \nnorm{u-w_h}+ \nnorm{w_h-u_h}\\
\Rightarrow \nnorm{u-u_h} &\leq (1 + \frac{M}{\alpha_h})\nnorm{u-w_h}
\end{align*}
2003 haben Xu und Zikatanov gezeigt, dass man die '$1$' weglassen kann.
\end{bemerkung}
Unser diskretes Problem: $a_h (u_h- v_h) = f(v_h) \; \forall v_h \in V_h $, $u_h \in V_h$. 
In $V_h$ sei $\set{\phi_i}_{i = 1}^N$ Basis von $V_h$. Damit lässt sich $u_h \in V_h$ darstellen als
\begin{align*}
  u_h = \sum_{i = 1}^N u_i \phi_i
\end{align*}
und nun sind die Koeffizienten $u_i$ gesucht. Auf unser Problem angewandt:
\begin{align*}
&  a(\sum_{i = 1}^N u_i \phi_i), \phi_j) = f(\phi_j) \quad j=1, \dots, N\\
&  \sum_{i = 1}^N a( \phi_i), \phi_j)u_i = f(\phi_j) 
\end{align*}
ist Gleichungssystem der Dimension $N$, die Koeffizientenmatrix($a(\phi_i, \phi_j)$) heißt Steifigkeitsmatrix.

\begin{beispiel} Einfaches Beispiel zum Vergleich von Ritz, Galerkin Methode der gewichteten Residuen, Kollokation

Modellproblem:
  \begin{align*}
 &   -u'' = f\\
&u(0) = u(1) = 0
  \end{align*}
  \begin{enumerate}
  \item Ritz: 
    \begin{align*}
      J(v) = \frac 1 2 \int_0^1v'^2 - \int_0^1 f v \Rightarrow \min
    \end{align*}
    mit $V = H^1_0(\O)$, $V_h = {\phi_1}$, $\phi_1 = x(1-x)$ die Basisfunktion.
    \begin{align*}
      J(u_h) = \frac 1 2 \int_0^1((u_1x(1-x))')^2 - \int_0^1 f u_1x(1-x) \Rightarrow \min
    \end{align*}
  \item Galerkin
    \begin{align*}
      &\int_0^1u' v' = \int_0^1 f v\\
      &u_1\int_0^1((x(1-x))')^2 = \int_0^1fx(1-x)
\end{align*}
  \item Methode des gewichteten Residuums

$Lu = f$, $Lu-f =$ Residuum, für $u_h \in V_h$:
    \begin{align*}
      \int_0^1(L u_h- f)\phi_i = 0 \quad forall 1 
    \end{align*}
    Der Rest soll orthogonal zum Residuum sein.
    \begin{align*}
      \int_0^1 (-(u_1x(1-x))'' -f)x(1-x) = 0
    \end{align*}
    Das ist identisch zur Gleichung des Galerkin-Verfahrens. Dises Übereinstimmung gibt es für genügend glatte Basisfunktionen. 
  \item Kollokation

    Das geht so: $Lu = f$ für $v \in V$, $V_h \subset V$, $\dim V_n = N$, 
    \begin{align*}
      (Lu_h - f)(\xi_i) = 0, \quad i = 1, \dots, N
    \end{align*}
    mit den Kollokationsstellen $\xi_i$. Diese Technik hat mit den anderen dreien nichts zu tun, die untereinander verwandt sind. Sie wird gerne auf Randproblemen angewand.
    Die Fehleranalyse für Kollokation ist allerdings schwierig.
  \end{enumerate}  
\end{beispiel}

\paragraph{Wie wählt man $V_h$?}
\begin{beispiel}
  \begin{align*}
    -u''= f, \qquad u(0) = u'(1) = 0
  \end{align*}
mit dem Raum $V = \set{v \in H^1: v(0) = 0}$ mit der Bilinearform $a(u,v) = \int_0^1 u'v'$.
Betrachte den Teilraum $V_h = \set{\frac{x^k} k, k = 1, \dots, N}$, 
\begin{align*}
  a(\phi_k, \phi_l) = \int_0^1 x^{k-1}x^{l-1} = \frac 1 {k+ l-1}
\end{align*}
Damit ergibt sich für die Koeffizientenmatrix
\begin{align*}
  \begin{pmatrix}
    1 &\frac{1}{2} &\frac{1}{3} & \dots \\
\frac{1}{2} &\frac{1}{3}\\
\frac{1}{3} & \vdots\\
\vdots 
  \end{pmatrix}
\end{align*}
Das ist die Hilbertmatrix mit einer extrem großen Kondition. Man kann das erzeugte Gleichungssystem für große $N$ nicht lösen.
\end{beispiel}
\begin{beispiel}
 Betrachte $-u'' = f$, $u(0)= u(1) = 0$, $V= H_0^1(0,1)$, $V_h = \set{\sin (j \pi x)}_{j = 1, \dots, N}$.
 \begin{align*}
   a(\phi_k, \phi_l)= \int_0^1 k \pi \cos (k \pi x) l \pi \cos(l \pi x) = 0 \quad k \neq l
 \end{align*}
Toll! Die Koeffizientenmatrix ist eine Diagonalmatrix!
\end{beispiel}
Wie wählt man also die Basisfunktionen und $V_h$?
\begin{itemize}
\item Sind die Approximationseigenschaften ok?
\item Sind die erzeugten Gleichungssystme lösbar?
\end{itemize}
Folgende Varianten sind beliebt: 
\begin{itemize}
\item Orthogonale Polynome (Legendre, trigonometrische Polynome, Gauß-Jacobi...) $\to$ spektrale Methoden
\item Splines $\to$ Finite Elemente (unsere Wahl)
\end{itemize}
