\section{Lösen von PDGLs mittels $\S\mO\F\E$}
\subsection{Poissongleichung und Behandlung von Randbedinungen}
\begin{align*}
  - a \Delta u &= f\\
\left.u\right|_{\Gamma_{1}}&= w\\
\left.\partial_{n}^{a} u\right|_{\Gamma_{2}}&= g\\
\left.\partial_{n}^{a} u\right|_{\Gamma_{3}}&= r u + s
\end{align*}
mit
\begin{align*}
  \partial^{a}_{n} u = a n \cdot\nabla u
\end{align*}
Gesucht ist ein $z \in H^{1}_{oD}}$, sodass $a(w + z, v) = l(v)$ für alle $ v\in H^{1}_{oD}$
\begin{align*}
  a(z, v) = \tilde l(v) - a(w, v) \eqqcolon l(v) \quad \forall v \in H^{1}_{oD}(\Omega). 
\end{align*}
Praktisch: betrachte $z \in H_{0}^{1}(\Omega)$: $a(z, v) =l(v)$ für alle $v \in H^{1}(\Omega)$ + streiche Spalte und Zeile zu den Dirichlet-Rand-Freiheitsgraden. 