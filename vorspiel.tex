\setcounter{page}{1}
\begin{itemize}
\item Prüfungen der beiden Vorlesungen getrennt
\item Aufgaben bei Herrn Schopf auf der Homepage
\item Buch: Numerische Behandlung partieller Differentialgleichungen (Vorlesung orientiert sich stark an diesem)
\item Buch: Die Finite-Elemente-Methode für Anfänger (sehr praktisch orientiert)
\end{itemize}
\section{Einige grundlegende Typen partieller DGL}
\begin{beispiel} Navier-Stokes-Gleichungen
  \begin{align*}
    u_t + u \nabla u + \nabla p - \nu \Delta u &= f \\
    \div u &= 0 \text{ + Randbedingungen}
  \end{align*}
  mit dem Geschwindigkeitsvektor $u$ und Druck $p$.
\end{beispiel}
Zunächst einmal sollte man numerisch lösen können:
\begin{enumerate}
\item Randwertproblem für Poissongleichung (elliptisch)
  \begin{align*}
    - \Delta u &= f \text{ in $\Omega$}\\
    u &= 0 \text{ auf } \partial \Omega
  \end{align*}
\item Wärmeleitungsgleichung (parabolisch)
  \begin{align*}
    u_t - \Delta u = f \text{ in $(0,T) \times \Omega$}
  \end{align*}
mit Anfangs- und Randbedingungen .
\end{enumerate}
\begin{beispiel} Black-Scholes-Gleichung (entartende parabolisch Gleichung)
  \begin{align*}
    v_t + 0.5 \sigma^2 s^2 V_{ss} + (r - \delta)s V_s -r V = 0
  \end{align*}
  
\end{beispiel}
